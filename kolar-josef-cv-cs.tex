% FortySecondsCV LaTeX template
% Copyright © 2019-2020 René Wirnata <rene.wirnata@pandascience.net>
% Licensed under the 3-Clause BSD License. See LICENSE file for details.
%
% Please visit https://github.com/PandaScience/FortySecondsCV for the most
% recent version! For bugs or feature requests, please open a new issue on
% github.
%
% Contributors
% ------------
% * ifokkema
% * Bertbk
% * Hespe
% * esben
%
% Attributions
% ------------
% * fortysecondscv is based on the twentysecondcv class by Carmine Spagnuolo
%   (cspagnuolo@unisa.it), released under the MIT license and available under
%   https://github.com/spagnuolocarmine/TwentySecondsCurriculumVitae-LaTex
% * further attributions are indicated immediately before corresponding code


%-------------------------------------------------------------------------------
%                             ADDITIONAL PACKAGES
%-------------------------------------------------------------------------------
\documentclass[
	a4paper,
	% showframes,
	% vline=2.2em,
	maincolor=cvviolet,
	% sidecolor=gray!50,
	sectioncolor=cvviolet,
	subsectioncolor=cvviolet,
	% itemtextcolor=black!80,
	% sidebarwidth=0.4\paperwidth,
	% topbottommargin=0.03\paperheight,
	% leftrightmargin=20pt,
	% profilepicsize=4.5cm,
	profilepicborderwidth=0pt,
	% profilepicstyle=profilecircle,
	% profilepiczoom=1.0,
	% profilepicxshift=0mm,
	% profilepicyshift=0mm,
	profilepicrounding=10mm,
	% logowidth=4.5cm,
	% logospace=5mm,
	% logoposition=before,
	% sidebarplacement=right,
]{fortysecondscv}

% fine tune line spacing
% \usepackage{setspace}
% \setstretch{1.1}

% improve word spacing and hyphenation
\usepackage{microtype}
\usepackage{ragged2e}

\usepackage{hyperref}
\hypersetup{
    colorlinks = true,
    urlcolor = cvviolet
}

% uncomment in case you don't want any hyphenation
% \usepackage[none]{hyphenat}

% take care of proper font encoding
\ifxetexorluatex
	\usepackage{fontspec}
	\defaultfontfeatures{Ligatures=TeX}
	% \newfontfamily\headingfont[Path=fonts/]{segoeuib.ttf} % use local font
\else
	\usepackage[utf8]{inputenc}
	\usepackage[T1]{fontenc}
\fi

% use a sans serif font as default
\usepackage[sfdefault]{ClearSans}
% \usepackage[sfdefault]{noto}
% \usepackage{lmodern}

% multi-language CV XeLaTeX and polyglossia (should also work with LuaLaTeX)
% NOTE: breaks \pointskill, \membership and some spacings
% \ifxetexorluatex
% 	\usepackage{polyglossia}
% 	\newfontfamily\arabicfontsf[Script=Arabic,Scale=1.5]{Amiri}
% 	\newfontfamily\englishfontsf{Clear Sans}
% 	\setmainfont{Amiri}
% 	\setdefaultlanguage{arabic}
% 	\setotherlanguage{english}
% \fi

% enable mathematical syntax for some symbols like \varnothing
\usepackage{amssymb}

% bubble diagram configuration
\usepackage{smartdiagram}
\smartdiagramset{
	% default font size is \large, so adjust to harmonize with sidebar layout
	bubble center node font = \footnotesize,
	bubble node font = \footnotesize,
	% default: 4cm/2.5cm; make minimum diameter relative to sidebar size
	bubble center node size = 0.4\sidebartextwidth,
	bubble node size = 0.25\sidebartextwidth,
	distance center/other bubbles = 1.5em,
	% set center bubble color
	bubble center node color = maincolor!70,
	% define the list of colors usable in the diagram
	set color list = {maincolor!10, maincolor!40,
	maincolor!20, maincolor!60, maincolor!35},
	% sets the opacity at which the bubbles are shown
	bubble fill opacity = 0.8,
}

\usepackage{enumitem}
\setlist[itemize]{leftmargin=0pt}

% CUSTOM
\graphicspath{{flags/flags/flags-iso/shiny/64/}{pics/}}

\usepackage[czech]{babel}

%-------------------------------------------------------------------------------
%                            PERSONAL INFORMATION
%-------------------------------------------------------------------------------
%% mandatory information
% your name
\cvname{Josef Kolář}
% job title/career
\cvjobtitle{Full Stack web vývojář}

%% optional information
% profile picture
\cvprofilepic{pics/profile.jpg}
% logo picture
% \cvlogopic{pics/profile.jpg}

% NOTE: ordering in sidebar will mimic the following order
\cvaddress{Litovel / Brno}
\cvsite{https://josefkolar.cz}
\cvmail{mail@josefkolar.cz}
%\cvkey{4096R/FF00FF00}{0xAABBCCDDFF00FF00}
\cvcustomdata{\faGithub}{\href{https://github.com/thejoeejoee}{@thejoeejoee}}
\cvphone{+420 777 951 637}

%-------------------------------------------------------------------------------
%                              SIDEBAR 1st PAGE
%-------------------------------------------------------------------------------

\addtofrontsidebar{
	\profilesection{Primárně}
		\pointskill{\faLaptopCode}{Python, Django}{5}[6]
		\pointskill{\faCode}{Vue.js, Nuxt.js}{4}[6]
		\pointskill{\faFileCode}{Typescript, ES6}{4}[6]
		\pointskill{\faCompress}{CSS, Tailwind.css}{4}[6]
		\pointskill{\faDatabase}{PostgreSQL, MySQL}{3}[6]

	\profilesection{Sekundárně}
		\skill{\faServer}{Linux, Docker, Webpack}
		\skill{\faLaptop}{C/C++, PHP, Nette}
		\skill{\faProjectDiagram}{{\large \LaTeX}, Git, VHDL}

	\profilesection{Jazyky}
		\pointskill{\flag{CZ.png}}{Čeština}{6}[6]
		\pointskill{\flag{GB.png}}{Angličtina}{4}[6]
		\pointskill{\flag{LV.png}}{Lotyština}{1}[6]
		\pointskill{\flag{DE.png}}{Němčina}{1}[6]

	\profilesection{O mně}
		\aboutme{
			Jsem komunikativní a extrovertní jedinec, který rád řeší problémy před něj postavené, zajímá se o své okolí a udržitelnost.
			Jako dobrovolník se věnuji práci s dětmi a mládeží, organizaci tábora
			či zahraničním studentům. \\Ve volném čase si rád pustím dokumentární filmy,
			studuju kryptoměny nebo jdu sportovat nebo fotit.
		}
}

%-------------------------------------------------------------------------------
%                         TABLE ENTRIES RIGHT COLUMN
%-------------------------------------------------------------------------------
\begin{document}

\makefrontsidebar

% {\large{Jsem zapálený Python vývojář s přesahem do frontend technologií.}}

% \vspace{-\baselineskip}
\cvsection{Pracovní zkušenosti}

\begin{cvtable}[2]
	\cvitem{2010 -- dosud}{Freelancer}{}{
	\vspace{-\baselineskip}
		\begin{itemize}
		\setlength\itemsep{0pt}
			\item prezentační weby a malé informační systémy
			\item Django, Vue.js, Bootstrap
			\item projekty jako
			\href{https://na.brnodobry.cz/}{NA Brno dobrý 2022},
			\href{https://prace.spseol.cz/}{Evidence závěrečných prací Thesaurus},
			\href{https://dortyodveverky.cz/}{Dorty od Veverky},
			\href{https://roveda.cz/}{Roveda s.r.o.},
			\href{https://hornizleb.cz/}{Myslivecký informační systém Horní Žleb},
			\href{https://cadoktor.cz/}{CADoktor}
		\end{itemize}
	}
	\cvitem{2014 -- 2019}{Junior full stack developer}{\href{https://www.olc.cz}{OLC Systems s.r.o.}}{
	\vspace{-\baselineskip}
	\begin{itemize}
		\setlength\itemsep{0pt}
		\item vývoj interního komponentového frameworku
		\item vývoj informačních (matričních a řídících) systémů
		\item práce v malém týmu vývojářů pro společnosti Český florbal, Mobilní pohotovost, Česká jezdecká federace či Spro
		\item Python, Django, PostgreSQL, Angular, Bootstrap CSS
		\item postupná migrace na Typescript, Vue.js a Webpack
	\end{itemize}
	}

\end{cvtable}

\cvsection{Vzdělání}
\begin{cvtable}[2]
	\cvitem{2016 -- 2022}{Fakulta informačních technologií}{VUT v Brně}{
	\vspace{-\baselineskip}
	\begin{description}
		\setlength\itemsep{0pt}
		\item[Magisterské studium] Informační technologie a umělá inteligence -- Počítačové sítě (nedokončeno)
		\item[ERASMUS+] {\small(08/2019-06/2020)}\\ University of Latvia, Faculty of Computing
		\item[Bakalářské studium] Informační technologie -- bakalářská práce \href{https://www.fit.vut.cz/study/thesis/21632/.cs}{Koordinace IoT na bázi MicroPythonu pomocí Node-RED}
	\end{description}
	}
	\cvitem{2012 -- 2016}{Střední průmyslová škola elektrotechnická}{Olomouc}
		{Elektrotechnika, dlouhodobá maturitní práce: \href{https://github.com/spseol/pybots-server/releases/download/v1.0/docs.pdf}{Souboje virtuálních robotů}}
\end{cvtable}

\cvsection{Dobrovolnická činnost}
\begin{cvtable}[2]
	\vspace{-\baselineskip}
	\cvitem{2020 -- dosud}{ESN VUT Brno, ESN Riga}{}{
	\begin{description}
		\vspace{-\baselineskip}
		\setlength\itemsep{0pt}
		\item[IT Coordinator] {\small(09/20-)} -- správa webových stránek, vývoj nástrojů pro automatizaci procesů, servis a vývoj IS pro párování studentů
		\item[Multimedia Manager] {\small(09/20-)} -- foto a video reportáže z akcí včetně jejich produkce
		\item[Events Manager] {\small(06/21-05/22)} -- v rámci vedení sekce jsem vedl tým, který plánoval a realizoval pravidelné akce pro zahraniční studenty VUT
		\item[Knowledge Manager] {\small(09/20-05/21)} -- správa a sběr znalostí a reportů na sekčním Google Drive disku
	\end{description}
	}
	\vspace{-\baselineskip}
	\cvitem{2017 -- dosud}{\href{https://rytirskytabor.cz/}{Rytířský tábor} -- hlavní organizátor}{Sarkander, z.s.}{
	\vspace{-\baselineskip}
	\begin{itemize}
		\setlength\itemsep{0pt}
		\item člen užšího vedení zodpovědný za vnější komunikaci a přihlášky
		\item spoluautor příběhu a táborového programu
		\item foto-video reportáž z tábora
	\end{itemize}
	}
	\cvitem{2013 -- 2019}{\href{https://minicup-archiv.tatranlitovel.cz/}{Litovel MINICUP} -- organizátor}{TJ Tatran Litovel}{
	\vspace{-\baselineskip}
	\begin{itemize}
		\setlength\itemsep{0pt}
		\item vedení technického týmu v rámci turnaje v miniházené
		\item realizace webových stránek a přímých přenosů,
		\item správa sociálních sítí
	\end{itemize}
	}
	%	\cvitem{12/2019 -- 06/2020}{Fotograf}{ESN Riga}{
	%		Na zahraničním výjezdu v Rize jsem se připojil ke studentské
	%		organizaci a pomáhal pořádat akce pro přijíždějící studenty,
	%		ze kterých jsem následně dělal i fotoreportáže.
	%	}
	%	\cvitem{08/2015 -- 06/2016}{Trenér házené -- mládež}{TJ Tatran Litovel}{
	%		Poznatky ze své hráčské a rozhodcovské kariéry jsem po celou sezónu předával mladým házenkářům ve věku 6-8 let.
	%	}
\end{cvtable}

% \cvsignature

\end{document}


%-------------------------------------------------------------------------------
%                              SIDEBAR 2nd PAGE
%-------------------------------------------------------------------------------
\addtobacksidebar{
\profilesection{About Me}
\aboutme{
The giant panda is a terrestrial animal and primarily spends its life
roaming and feeding in the bamboo forests of the Qinling Mountains and in
the hilly province of Sichuan.
}
\profilesection{Soft Skills}
\pointskill{\faHome}{Looking Cute}{4}[4]
\skill[1.8em]{\faCompress}{No need to specify further}
\pointskill{\faChild}{Chillin' hard}{3}[4]
\skill[1.8em]{\faCompress}{On a tree}
\skill[1.8em]{\faCompress}{In the grass}

\profilesection{Diagrams}
\begin{sidebarminipage}
	\chartlabel{Bubble}
	\chartlabel{Diagrams}
	\chartlabel{with}
	\chartlabel{proper}
	\chartlabel{overflow}
	\chartlabel{protection}
	\chartlabel{for}
	\chartlabel{labels}
\end{sidebarminipage}

\begin{figure}\centering
\smartdiagram[bubble diagram]{
\textcolor{white}{\textbf{Being a}} \\
\textcolor{white}{\textbf{Panda}}, % center bubble
\textcolor{black!90}{Eating},
\textcolor{black!90}{Sleeping},
\textcolor{black!90}{Rolling},
\textcolor{black!90}{Playing},
\textcolor{black!90}{Chilling}
}
\end{figure}

\chartlabel{Wheel Chart}

\wheelchart{3.7em}{2em}{%
20/3em/maincolor!50/Chill,
15/3em/maincolor!15/Play,
30/4em/maincolor!40/Sleep,
20/3em/maincolor!20/Eat
}

\profilesection{Barskills}
\barskill{\faSkyatlas}{Wearing asian rice hats}{60}
\barskill{\faImage}{Playing Chess}{30}
\barskill{\faMusic}{Playing the bamboo flute}{50}

\profilesection{Memberships}
\begin{memberships}
	% \membership[4em]{pics/logo.png}{PandaScience.net}
	% \membership[4em]{pics/logo.png}{Some longer text spanning over more than only one line}
\end{memberships}
}


\newpage
\makebacksidebar
% \newgeometry{
% 	top=\topbottommargin,
% 	bottom=\topbottommargin,
% 	right=\leftrightmargin,
% 	left=\leftrightmargin
% }

\cvsection{section}
\cvsubsection{Subsection}
\begin{cvtable}
	\cvitem{<dates>}{<cv-item title>}{<location>}{<optional: description>}
\end{cvtable}

\cvsection{cvitem}
\cvsubsection{Multi-line with longer description}
\begin{cvtable}
	\cvitem{date}{Description}{location}{Some longer and more detailed
		description, that takes two lines of space instead of only one.}
	\cvitem{date}{Description}{location}{Some longer and more detailed
		description, that takes two lines of space instead of only one.}
	\cvitem{date}{Description}{location}{Some longer and more detailed
		description, that takes two lines of space instead of only one.}
\end{cvtable}

\cvsubsection{One-line without description}
\begin{cvtable}
	\cvitem{Award}{One-line description}{Sponsor}{}
	\cvitem{Award}{One-line description}{Sponsor}{}
	\cvitem{Award}{One-line description}{Sponsor}{}
\end{cvtable}

\cvsection{cvitemshort}
\cvsubsection{One-line}
\begin{cvtable}
	\cvitemshort{Key}{Some further description}
	\cvitemshort{Key}{Some further description}
	\cvitemshort{Key}{Some further description}
\end{cvtable}

\cvsubsection{Multi-line with longer description}
\begin{cvtable}
	\cvitemshort{Key}{Some further description. Can fill even more than
		only one single line while still keeping the correct indendation level.}
	\cvitemshort{Key}{Some further description. Can fill even more than
		only one single line while still keeping the correct indendation level.}
	\cvitemshort{Key}{Some further description. Can fill even more than
		only one single line while still keeping the correct indendation level.}
\end{cvtable}

\cvsection{cvpubitem}
\begin{cvtable}
	\cvpubitem{Publication title}{Authors}{Journal}{Year}
	\cvpubitem{Publication title}{Authors}{Journal}{Year}
	\cvpubitem{Publication title that is spanning over multiple lines and still
		does not look too bad}{Authors}{Journal}{Year}
\end{cvtable}


\end{document}
